%E Even page
%O Odd page
%L Left field
%C Center field
%R Right field
%H Header
%F Footer
\documentclass[a4paper]{article}
\usepackage[utf8]{inputenc}
\usepackage[T1]{fontenc}
\usepackage{lmodern}
\usepackage{ngerman}
\usepackage{fancyhdr}
\usepackage{multicol}
\pagestyle{fancy}
\fancyhead{}
\fancyfoot{}
\fancyhead[L]{Latein-Übersetzungen}  %Kopfzeile links			<Titel>
\fancyhead[C]{L1 - BKS}%Mitte Kopfzeile			<Fach>  BKS
\fancyhead[R]{\today} %Rechts Kopfzeile   	<Datum>
\fancyfoot[L] {\thepage} %Fusszeile Links mit Seitenzahl
\fancyfoot[R] {nc et al.} %Fusszeile Rechts mit Name
\begin{document}
\title{Latein - Übersetzungen der Unterrichtstexte \\
	\author{nc et al.}}
	\maketitle
\tableofcontents
\newpage
\section{Gaudeamus Igitur}
\begin{multicols}{2}
	Gaudeamus igitur, \\
	iuvenes dum sumus; \\
	Post iucundam iuventutem, \\
	Post molestam senectutem \\
	nos habebit humus! \\

	Vita nostra brevis est, \\
	brevi finietur, \\
	Venit mors velociter, \\
	rapit nos atrociter, \\
	nemini parcetur. \\

	Ubi sunt, qui ante \\
	nos in mundo fuere? \\
	Vadite ad superos, \\
	transite ad inferos, \\
	ubi iam fuere. \\

	Vivat academia, \\
	vivant professores, \\
	vivat membrum quodlibet, \\
	vivant membra quaelibet, \\
	semper sint in flore! \\

	Vivant omnes virgines \\
	faciles, formosae, \\
	vivant et mulieres, \\
	tenerae, amabiles, \\
	bonae, laboriosae! \\

	Vivat et respublica \\
	et qui illam regit, \\
	vivat nostra civitas, \\
	Maecenatum caritas, \\
	quae nos hic protegit! \\

	Pereat tristitia, \\
	pereant osores, \\
	pereat diabolus, \\
	quivis antiburschius, \\
	atque irrisores! \\


	Darum lasst uns fröhlich sein, \\
	solange wir noch jung sind! \\
	Nach einer angenehmen Jugend,  \\
	nach einem beschwerlichen Alter \\
	wird uns das Grab haben. \\

	Unser Leben ist kurz, \\
	in Kürze wird es beendet werden. \\
	Der Tod kommt schnell, \\
	er rafft uns grausam dahin, \\
	niemand wird verschont werden. \\

	Wo sind die, die vor \\
	uns auf der Welt waren? \\
	Geht in den Himmel, \\
	geht hinüber in die Hölle, \\
	wo sie schon waren. \\

	Es lebe die Universität, \\
	es leben die Professoren, \\
	es lebe jedes Mitglied, \\
	es leben alle Mitglieder, \\
	mögen sie immer in Blüte stehen!\\

	Es leben alle jungen Fraeun, \\
	freundlich, wohlgeform, \\
	es leben auch die Ehefrauen, \\
	zart, liebenswert, \\
	gut, arbeitssam! \\

	Es lebe auch der Staat, \\
	und der, der jenen lenkt. \\
	Es lebe unser Bürgertum, \\
	die Hochachtung der Mäzene, \\
	die uns hier beschützt. \\

	Zugrundegehen soll die Traurigkeit, \\
	zugrundegehen sollen die Hasser, \\
	zugrundegehen soll der Teufel, \\
	jeder Feind der Burschen, \\
	und auch die Spötter!
\end{multicols}

\section{Fabeln des Phädrus}
\subsection{Das Recht des Stärkeren}
Angetrieben vom Durst waren ein Wolf und ein Lamm zum selben Bach gekommen. Weiter oben stand der Wolf, recht viel weiter unten das Lamm. Dann suchte der Räuber, angetrieben von massloser Fresssucht, einen Grund zum Streit.

,,Warum'', sagte er, ,,hast du mir das Wasser trüb gemacht, während ich trank?''
Das wolletragende Tier erwiderte ängstlich: ,,Wie kann ich, bitte, das tun, das du beklagt hast, Wolf? Von dir fliesst das Wasser zu meinen Schlücken herab!''
Widerlegt durch die Kräfte der Wahrheit, sagte jener: ,,Zuvor hast du sechs Monate lang mir gelästert!'' Das Lamm antwortete: ,,Ich für meinen Teil war dann noch nicht geboren!'' ,,Beim Herkules, den Vater hat über mich gelästert!'', und so zerfleischt er es, nachdem er es gepackt hatte, wobei der Tod ungerecht war.
Diese Fabel wurde geschrieben wegen jenen Menschen, die Unschuldige unter vorgetäuschten Gründen unterdrücken.

\subsection{Bestrafte Eitelkeit}
Wer sich darüber freut, mit hinterlistigen Worten gelobt zu werden, zahlt heftige Strafe aufgrund der späten Reue.

Als ein Rabe ein vom Fenster gestohlenes Stück Käse fressen wollte, wobei er sich auf einem hohen Baum niederliess, sah ein Fuchs diesen, dann begann er [der Fuchs] so zu sprechen:
,,Oh welcher Glanz deine Federn haben, Rabe! Wie viel Schmuck du am Körper und im Gesicht trägst! Wenn du eine Stimme hättest, wäre kein Vogel dir überlegen!''
Aber jener Dummkopf, während er seine Stimme vorführen wollte, liess das Stück Käse aus dem Mund fallen, welches die listige Füchsin schnell gierig mit den Zähnen raubte.

Dann schliesslich seufzte die getäuschte Dummheit des Raben auf.

\subsection{Wenn die Trauben zu hoch hängen}
Ein vom Hunger getriebener Fuchs versuchte nach einer Traube in einem hohen Weinstock zu greifen, wobei er mit höchsten Kräften danach sprang. Als er diese nicht berühren konnte, sagte er im Weggehen: ,,Sie ist noch nicht reif, ich will keine Bittere nehmen!''

Diejenigen, die das was sie nicht tun können, mit Worten beschönigen, werden dieses Beispiel auf sich beziehen müssen.

\subsection{Gerechte Aufteilung}
Niemals gibt es treue Freundschaft mit einem Mächtigen. Diese Fabel beweist meine These:

Eine Kuh, eine Ziege und ein Schaf, das gewohnt war, Unrecht zu ertragen, waren Gefährten mit einem Löwen im Wald. Als diese einen Hirschen mit einem gewaltigen Körper erlegt hatten, sprach der Löwe, nachdem er die Teile gemacht hatte, folgendermassen:

,,Ich nehme den ersten [Teil], weil ich Löwe heisse; den zweiten werdet ihr mir geben, weil ich Teilhaber bin, dann wird mir der dritte gebühren, weil ich ich stärker bin. Wenn jemand den vierten berührt, wird er bestraft.''

So trug die reine Unverschämtheit die ganze Beute weg.

\subsection{Vor Neid zerplatzt}
Der Schwache geht zugrunde, wenn er den Starken nachahmen will.

Auf einer Wiese erblickte einst ein Frosch ein Rind und blies, berührt von der Eifersucht auf so eine beachtliche Grösse, seine runzlige Haut auf. Dann fragte er seine Kinder, ob er grösser als das Rind sei. Sie verneinten. Wiederum spannte er seine Haut mit grösserer Anstrengung auf, und fragt auf ähnliche Weise, wer grösser sei. Jene sagten ,,das Rind''.

Schliesslich lag er, als er sich stärker aufblasen wollte, mit zerrissenem Körper da.

\subsection{Die verkaterte Maus}
Einst fiel eine Maus in ein Wein- oder Bierfass. Ein vobeikommender Kater hörte die Maus jämmerlich piepsen, weil sie nicht hinaus kommen konnte. Und der Kater sagt: ,,Warum klagst du?'' Die Maus antwortet: ,,Weil ich hier nicht herauskommen kann'' Der Kater: ,,Was wirst du mir geben, wenn ich dich herausziehe?'' Die Maus: ,,Alles was du verlangst.'' Und der Kater sagte: ,,Wenn ich dich dieses Mal befreie, wirst du zu mir kommen, wenn ich dich rufe?'' Und die Maus sagte: ,,Das verspreche ich fest.'' Der Kater: ,,Schwöre mir!'' Und die Maus schwor. Der Kater zog die Maus heraus und erlaubte ihr zu gehen.

Einmal war der Kater hungrig und kam zum Mausloch und befahl ihr, hinauszukommen. Die Maus sagt: ,,Das werde ich nicht tun!'' Der Kater erzürnt: ,,Hast du es mir nicht geschworen?'' Die Maus antwortet: ,,Bruder, ich war betrunken, als ich geschworen habe!''

\section{Inhaltsangeben des Hygin}
\subsection{Hygin 153 - Deukalion und Pyrrha}
Als die Sintflut, die wir Überschwemmung oder Überflutung nennen, kam, ging das ganze Menschengeschlecht unter, ausser Deukalion und Pyrrha, die auf den Berg Ätna flohen, der der höchste auf Sizilien sein soll.

Weil diese wegen der Einsamkeit nicht leben konnten, erbten sie von Jupiter, dass er entweder ihnen Menschen gibt, oder ihnen das gleiche Unglück zuteil werden lasse. Dann befahl Jupiter ihnen, Steine hinter sich zu werfen, denen, die Deukalion warf, befahl er, Männer zu sein, denen die Pyrrha warf, Frauen. Wegen dieses Umstandes heisst Volk ,,laos'', denn ,,laos'' heisst auf Griechisch Stein.
\subsection{Hygin 146 - Dädalus und Ikarus}
Dädalus, der Sohn des Eupalamus, der das Handwerk von Minerva empfangen haben soll, warf Perdix, den Sohn seiner Schwester, wegen des Neides auf ein Kunstwerk (weil dieser nämlich zuerst die Säge erfunden hatte), von einer Dachspitze.

Wegen dieses Verbrechens ging er weg von Athen nach Kreta zu König Minos ins Exil.

\section{Ovid. Metamorphosen}
\subsection{Deukalion und Pyrrha}
\textit{Ovid beschreibt die ,,verkehrte Welt'', die durch die Flut einstanden ist: }

Und schon bestand zwischen Meer und Erde kein Unterschied: Alles war Meer, auch fehlten dem Meer die Küsten. Dieser besetzt einen Hügel, der andere sitzt in einem gebogenen Boot und zieht die Ruder dort, wo er neulich pflügte: Jener segelt über Saatfelder oder die Dächer eines versunkenen Landhauses, dieser fängt einen Fisch auf der Spitze einer Ulme. Der Anker haftet, wenn es der Zufall wollte, an einer grünen Wiese, oder die gekrümmten Kiele streifen unterhalb liegende Weingärten: und wo vor kurzem hagere Schafe das Gras abgefressen haben, dort legen nun die Robben ihre unförmigen Körper nieder. Die Nereiden wundern sich unter Wasser über die Haine, Städte und Häuser, die Delphine bewohnen die Wälder, stoßen an tiefhängende Äste und schlagen an die heftig bewegten (schwankenden) Stämme. Der Wolf schwimmt unter den Schafen, eine Woge trägt die blonden Löwen mit sich fort, eine Woge die Tiger; weder nützen dem Eber die Kräfte seiner Hauer, noch dem fortgerissenen Hirschen die schnellen Schenkel (Beine).

\textit{Schliesslich lässt Jupiter die Fluten wieder zurücktreten. Deukalion, der Sohn des Prometheus, wünscht sich nur eines: wie sein Vater Menschen erschaffen zu können:}

,,Oh, wenn ich doch die Völker mit den väterlichen Künsten wiederherstellen und der geformten Erde Leben einhauchen könnte! Nun bleibt das menschliche Geschlecht in uns beiden übrig. So gefiel es den Göttern. Und als die (einzigen) Vertreter der Menschen bleiben wir.'' Er hatte es gesagt, und sie weinten: Man beschloß, den göttlichen Willen anzubeten und Hilfe durch heilige Orakelsprüche zu suchen.

\textit{Die beiden begeben sich zu Themis, der Göttin der Weissagung, und erhalten folgende Auskunft:}

Die Göttin wurde bewegt und gab einen Orakelspruch: ,,Verlaßt den Tempel, verhüllt das Haupt, löst die gegürteten Kleider und werft Knochen der großen Mutter hinter den Rücken!''

\textit{Sie sollen Steine hinter den Rücken werfen, um Menschen zu erschaffen. }
\subsection{Dädalus und Ikarus}
\textit{Für König Minos wird Dädalus zum unentbehrlichen Helfer - unter anderem erbaut er auf Kreta das berühmte Labyrinth. Als er in die Heimat zurückkehren will, lässt Minos ihn nicht ziehen: }

Dädalus, der inzwischen Kreta sowie die lange Verbannung sehr haßte und von der Liebe zur Heimat berührt war, war vom Meer abgeschnitten. ,,Wenn er auch das Land und das Meer versperrt: Aber der Himmel steht sicherlich offen. Dorthin werden wir gehen. Alles mag er besitzen, die Luft (aber) besitzt Minos nicht!'' Er sprach es, versenkte seinen Geist in unbekannte Künste und veränderte die Natur.

\textit{Dädalus fügt Vogelfedern aneinander und verbindet sie mit Wachs und Faden, sodass sie ganz den Anschein von echten Flügeln erwecken:}

Der Knabe Icarus stand dabei und griff bald, ohne zu wissen, daß er mit seiner Gefahr spielte, mit lächelndem Mund nach den Federn, welche die wehende Luft bewegt hatte, machte bald das gelbe Wachs mit dem Daumen weich und behinderte durch sein Spiel das erstaunliche Werk seines Vaters. Nachdem letzte Hand an das Werk angelegt worden war, schwang der Erbauer selbst seinen Körper in die beiden Flügel und schwebte in der bewegten Luft. Er unterwies auch seinen Sohn und sagte: ,,Daß Du auf der mittleren Bahn fliegst, mahne ich Dich, Ikarus, damit nicht, wenn du zu tief fliegst, das Wasser die Federn schwer macht, und, wenn Du zu hoch fliegst, die Glut (der Sonne) sie versengt. Fliege dazwischen!''

\textit{Von bösen Vorahnungen geplagt, gibt Dädalus seinem Sohn eine Kuss, dann steigen die beiden in die Lüfte:}

Und schon lag auf der linken Seite das Samos der Juno (sowohl Delos als auch Paros waren zurückgelassen = überflogen) und auf der rechten Seite Lebinthos und das an Honig reiche Calymne, als der Knabe begann, am kühnen Flug Freude zu haben, den Führer verließ und, fortgerissen durch die Begierde nach dem Himmel, einen höheren Weg einschlug. Die Nähe der brennden Sonne weicht das duftende Wachs, die Verbindung der Federn, auf. Das Wachs war geschmolzen: jener schüttelt die nackten Arme, bekommt ohne Ruder keine Luft zu fassen, und das den Namen des Vaters schreiende Gesicht wird vom blauen Wasser aufgenommen, welches von jenem den Namen bekam \textit{(die ikarische See)}. Aber der unglückliche Vater, jetzt aber kein Vater mehr, sagte ,,Ikarus'', ,,Icarus'' sagte er, ,,wo bist Du? In welcher Richtung soll ich Dich suchen?'' ,,Ikarus'' sagte er immer wieder: er erblickte die Federn im Meer, verwünschte seine Künste, barg den Leichnam in einem Grab, und die Landschaft wurde nach dem Namen des Bestatteten benannt.

\section{Vergil, Äneis}
\subsection{Die verlassene Dido}
Als die Königin vom Wachturm aus das erste Licht aufkommen, die Flotte mit gleichgerichteten Segeln auslaufen sah und die Küste und den Hafen leer ohne Ruderknechte bemerkte, schlug sie sich drei- und viermal mit der Hand auf ihre schöne Brust, raufte sich die blonden Haare und sagte: ,,Beim Jupiter! Dieser wird gehen und sollte der Fremdling mit meiner Herrschaft sein Spiel getrieben haben? Geht, bringt schnell Feuer herbei, gebt die Waffen aus, legt euch in die Ruder! Was rede ich? Oder wo bin ich? Welcher Wahnsinn verändert meinen Geist? Unglückliche Dido, sucht dich nun die ruchlose Tat heim?''






\end{document}
